\section{Task Description}
The goal of this task is to analyze a Turing activator-inhibitor dynamics on networks, with specific suggestion to replicate the findings of \cite{main_network}. 
\noindent
A system governed by a reaction-diffusion mechanism can exibit pattern emergence when perturbed from an initial linearly stable equilibrium state. The conditions for pattern initiation are found through linear stability analysis. The subsequent evolution of the pattern is non-linear and eventually results in a steady state (that can be either stationary or time dependent).
\\
\textbf{A note:} The authors of \cite{main_network} only state the conditions that must hold for pattern formation on the network. In fact, they can be derived with only a few minor changes in the same way as one does for pattern formation in a continuous medium. However, if the reader is not familiar with the traditional Turing patterns in continuous space (like me before this work), those relations are not at all evident. I wanted to get a true understanding of what I was going to simulate. That is why I studied in detail the case of the continuous medium and then I derived the conditions stated by the authors of \cite{main_network}. However, this meant that, to keep the workload manageable, I was able to focus only on the analysis of the \textbf{pattern initiation} stage and disregard the anaylis of the multistability and histeresis effects done in the second part of the article.
